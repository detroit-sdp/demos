\documentclass{beamer}
\usepackage[ruled,vlined]{algorithm2e}
\usetheme{Antibes}
\title{Tadashi}
\subtitle{Group 15, Demo 2}
\institute{\vspace{2em}\includegraphics[height=3cm]{figs/logo.png}}
\date{February 26, 2020}

\begin{document}
\maketitle
\section{Introduction}
\begin{frame}
  \frametitle{Tadashi: a robot for assisted living}
  \begin{itemize}
    \item Tadashi is a robot for assisted living and care homes.
    \item Tadashi delivers food and water to residents.
    \item Caregivers can schedule and track deliveries --- getting an insight into the nutrition and hydration of residents they're caring for.
  \end{itemize}
\end{frame}

\section{Robot movement}
\begin{frame}
  \frametitle{Live demo: receipt of a packet and moving the robot}
\end{frame}
\begin{frame}
  \frametitle{Live demo: local cost map}
\end{frame}

\section{Lift}
\begin{frame}
  \frametitle{Quantitative analysis: lift height}
  {\bf Goal: } allow residents to pick up food and water while sitting without needing to bend down.
  \begin{center}
    \includegraphics[width=4cm]{figs/Lift_down_height.png}
    \includegraphics[width=4cm]{figs/Lift_up_height.png}
  \end{center}
  $\implies$ sitting residents can reach tray without need to bend down. 
\end{frame}

\begin{frame}
  \frametitle{Quantitative analysis: lift speed}
  {\bf Goal: } estimate how long residents need to wait to pick up their delivery after the robot arrives.
  \begin{center}
    \includegraphics[width=6cm]{figs/Lift_speed.pdf}
  \end{center}
  $\implies$ reciprocal relation; use speed of 400 with time of 20.03s.
\end{frame}

\begin{frame}
  \frametitle{Quantitative analysis: lift load}
  {\bf Goal: } inform caregivers how much weight can safely be put on the lift.
  \begin{center}
    \begin{tabular}{llllr}
      {\bf Weight (g)} & {\bf Mean time to lift (s)} & {\bf \# tests} & {\bf \# successes}\\
      \hline
      79 & 39.62  & 10 & 10  \\
      613 & 39.72 & 10 & 10 \\
      1149 & 39.94 & 10 & 10  \\
      1326 & 40.19 & 10 & 10 \\
      1682 & 40.20 & 10 & 10 \\
      1858 & 39.64 & 10 & 10  \\
      2034 & 40.39 & 7 & 6
    \end{tabular}
  \end{center}
  $\implies$ do not use with weights of 2kg or above. 
\end{frame}

\begin{frame}
  \frametitle{Quantitative analysis: weight}
  {\bf Goal: } estimate how long residents will need to wait, depending on the weight on the lift.
  \begin{center}
    \includegraphics[width=9cm]{figs/Lift_time.pdf}
  \end{center}
  $\implies$ statistically significant increase in waiting time with heavier weights, but not noticeable to observer (difference of 0.77s).
\end{frame}

\section{App}
\begin{frame}
  \frametitle{Proposed dashboards}
  \begin{center}
    \includegraphics[width=3cm]{figs/prototype1.png}
    \includegraphics[width=3cm]{figs/prototype2.png}
    \includegraphics[width=3cm]{figs/prototype3.png}
  \end{center}
\end{frame}

\begin{frame}
  \frametitle{Main app screens}
\end{frame}

\section{Appendix}
\begin{frame}
  \frametitle{Statistical testing: difference between means}
  The mean time taken to lift 79g and 2.0kg is different. We can test this using a two-sample t-test with:
  $$H_0: \mu_1 = \mu_2,\ H_1: \mu_1 \neq \mu_2 $$
  $$ \mu_1 = 39.62,\ \mu_2 = 40.39,\ s^2_1 = 0.002,\ s^2_2 = 0.110,\ N_1 = 10,\ N_2 = 6$$
  Then we can calculate the test statistic $T$:
  $$T = \frac{\mu_1 - \mu_2}{\sqrt{s^2_1 / N_1 + s^2_2 / N_2}} = -5.66$$

  For a t-distribution with 4 degrees of freedom, assuming unequal variances and $p = 0.01$, we need $|T| > 3.75$. Since $|T| = 5.66$, $|T| > 3.65$ so we can reject the null hypothesis with $p < 0.01$.

  So the mean time taken to lift 79g and 2.0kg is different. 
\end{frame}

\begin{frame}
  \frametitle{Testing algorithm}
  \begin{algorithm}[H]
    \KwResult{n samples for each speed in $[200, 300, \ldots, 1000]$}
    \For{speed in $[200, 300, \ldots, 1000] $}{
      results = list of size $n$ \\
      \For{test in $[1, 2, \ldots, n]$}{
        start time = current time
        
        \While{lift is not at top}{
          raise the lift
        }
        
        end time = current time

        append to results (end time - start time)
        
        \While{lift is not at bottom}{
          lower the lift
        }

      }
    }
  \end{algorithm}
\end{frame}
\begin{frame}
  \frametitle{User survey questions (I)}
  {\bf Goal: } ensure app meets the needs and wants of caregivers.
  {\bf Approach: } perform interviews with working caregivers: 
  \begin{enumerate}
    \item[1] Talk me through your day-to-day tasks working in a care home.
    \item[2] Are there any jobs you can think of that need to be done but take a lot of time; time that you could be spending on more important tasks?
    \item[3] Are there any tasks that you think could be made more efficient if there was someone with no previous care home experience/qualifications there to help you?
    \item[4]  Is there a lot of paperwork/note taking required working as a carer in this care home environment? If so, what does this include?
  \end{enumerate}
\end{frame}

\begin{frame}
  \frametitle{User survey questions (II)}
  \begin{enumerate}
    \item[5] Would you say that you and the other carers have a mobile phone that you use every day, and can easily work?
    \item[6] Do you think that this robot would a helpful addition in the care home?
    \item[7] Do you think that using a robot in a care home would be welcomed by caregivers and residents alike? Would you say the robot could potentially scare the residents? If so, how?
    \item[8] What is your opinion on using a lift to raise the food/water to the resident?
    \item[9] Are there any problems that you think this robot would cause if used in the way we are planning?
    \item[10] Are there any other tasks that you think would be useful for a robot of this type to do?
  \end{enumerate}
\end{frame}

\begin{frame}
  \frametitle{User survey questions (III)}
  \begin{enumerate}
    \item[11] What features would you find useful to also be included in the dashboard app?
    \item[12]  What information about the residents do you have to keep track of everyday that require calculations that could be made more efficient by the app automatically calculating and evaluating?
  \end{enumerate}
\end{frame}
\end{document}
